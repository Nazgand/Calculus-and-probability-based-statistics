
\documentclass[8pt]{amsart}

\usepackage{amssymb}
\usepackage{amsmath}
\usepackage{graphics}

\newtheorem{thm}{Theorem}[section]
\newtheorem{prop}[thm]{Proposition}
\newtheorem{lem}[thm]{Lemma}
\newtheorem{cor}[thm]{Corollary}


\theoremstyle{definition}
\newtheorem{definition}[thm]{Definition}
\newtheorem{example}[thm]{Example}


\theoremstyle{remark}
\newtheorem{remark}[thm]{Remark}

\numberwithin{equation}{section}

\newcommand{\R}{\mathbb{R}} % The real numbers.
\newcommand{\dd}[1]{\mathrm{d}#1}

\begin{document}

\title{Calculus and probability based statistics}

\author{Mark Andrew Gerads}
\date{\today}
\address{
318 Thomas Drive\\
Marshalltown, Iowa, 50158
}
\email{MGerads11@winona.edu\\nazgand@gmail.com}
\maketitle
\section{Finite uniform statistics}
\subsection{Introduction}
The purpose of this text is to have statistics built on a solid foundation of probability and calculus.
\definition{$A$ is an independently repeatable event.}
\definition{$p\in[0,1]$ is the probability that $A$ will happen when $A$ is given a chance to happen. For simplicity, we assume that $p$ is the same every time $A$ is given a chance to happen.}
\definition{$n\in\mathbb{W}$ is the number of times $A$ had a chance to happen.}
\definition{$k\in\mathbb{W}$ is the number of times $A$ actually happened.}
\remark{$k\leq n$.}
\definition{$f_p$ is the prior probability distribution for $p$. In the uniform statistics section, we assume $f_p=1$.}
\remark{\label{IntegrateToUnity0}$1=\int_0^1 f_p \dd p$}
\definition{$f_{p|n,k}$ is the updated probability distribution for $p$.
\remark{\label{IntegrateToUnity1}$1=\int_0^1 f_{p|n,k} \dd p$}
\remark{$k/n$ is the expected probability of $p$ and is where the maximum value of the function $f_{p|n,k}$ is. }
\definition{$(a,b)$ is the confidence interval of $p$ such that $b-a$ is minimized.}
\definition{$c$ is the desired certainty for the confidence interval of $p$.}
\remark{$0\leq c=\int_a^b f_{p|n,k} \dd p\leq 1$}
\remark{$0\leq a\leq b\leq 1$}
\remark{The symmetry mentioned is where you swap $p$ with $1-p$ and $k$ with $n-k$ and $A$ with $\lnot A$.}

\subsection{The General Prior Probability Problem}
$n,k,c$ are known, and we must solve for $a,b$.
\remark{The probability of $k$ events given $n$ chances is $\binom{n}{k}p^k(1-p)^{n-k}$, but we must scale it to satisfy \ref{IntegrateToUnity1}:
\begin{equation}
f_{p|n,k}
=\frac{\binom{n}{k}p^k(1-p)^{n-k}}{\int\limits_{0}^{1}\binom{n}{k}p^k(1-p)^{n-k}\dd p}
=\frac{p^k(1-p)^{n-k}}{\int\limits_{0}^{1}p^k(1-p)^{n-k}\dd p}
=(n+1)\binom{n}{k}p^k(1-p)^{n-k}
\end{equation}
}
\subsection{The Uniform Prior Probability Problem}
$n,k,c$ are known, and we must solve for $a,b$.
\remark{The probability of $k$ events given $n$ chances is $\binom{n}{k}p^k(1-p)^{n-k}$, but we must scale it to satisfy \ref{IntegrateToUnity1}:
	\begin{equation}
	f_{p|n,k}
	=\frac{\binom{n}{k}p^k(1-p)^{n-k}}{\int\limits_{0}^{1}\binom{n}{k}p^k(1-p)^{n-k}\dd p}
	=\frac{p^k(1-p)^{n-k}}{\int\limits_{0}^{1}p^k(1-p)^{n-k}\dd p}
	=(n+1)\binom{n}{k}p^k(1-p)^{n-k}
	\end{equation}
}
\subsection{The 4 cases of finding the interval}

\subsubsection{Case $n=k=0$}
This case is can be self-symmetrical.
\remark{We have the general solution \begin{equation}
(a,b)=(a,a+0.5)
\end{equation}, but the self-symmetric solution is \begin{equation}
(a,b)=\left(\frac{1-c}{2},\frac{1+c}{2}\right)
\end{equation}}

\subsubsection{Case $n>k=0$}
\label{Symetry 1A}
This case is symmetrical to case \ref{Symetry 1B}.
\remark{We can solve \begin{equation}
\int_0^b (n+1)(1-p)^{n} \dd p=c=1-(1-b)^{n+1}
\end{equation}}
\remark{We have \begin{equation}
	(a,b)=\left(0,1-(1-c)^\frac{1}{n+1}\right)
	\end{equation}}
\example{
	Now we can analyze the old saying "Third time's the charm.". $n=2$.
	\begin{equation}
	c=0.5\Rightarrow b\approx 0.2062994740,
	c=0.95\Rightarrow b\approx 0.6315968501
	\end{equation}
	We can now see that if you fail at something twice, then success before the 8th try is worth approximately a coin flip. The expected time to find success can be calculated through this formula: $\left\lceil n+\frac{1}{b}\right\rceil$.
	}
\example{
	Suppose we were repeatedly trying a difficult task. How many times of trying without success must there be for us to conclude with $c=0.99$ that $b\leq 0.01$?
	\begin{equation}
	n=\frac{2 \log (3)+\log (11)}{2 \log (2)-2 \log (3)+2 \log (5)-\log (11)}
	\approx 457.2105766
	\end{equation}
	We can see "If at first you don't succeed, try, try again." is sensible for things that we want to be certain are improbable before giving up.
}

\subsubsection{Case $n=k>0$}
\label{Symetry 1B}
This case is symmetrical to case \ref{Symetry 1A}.
\remark{We can solve \begin{equation}
	\int_a^1 (k+1)p^k \dd p=c=1-a^{k+1}
	\end{equation}}
\remark{We have \begin{equation}
	(a,b)=\left((1-c)^\frac{1}{k+1},1\right)
	\end{equation}}

\subsubsection{Case $n>k>0$}
This case is self-symmetrical.
\remark{We desire to solve \begin{equation}
	\int_a^b (n+1)\binom{n}{k}p^k(1-p)^{n-k} \dd p=c=(n+1)\binom{n}{k}\left(B_b(k+1,-k+n+1)-B_a(k+1,-k+n+1)\right)
	\end{equation} where $B$ is the incomplete beta function.
}
\remark{
 We can use the hints
	\begin{equation}
	a^k(1-a)^{n-k}=b^k(1-b)^{n-k}
	\end{equation}
}
\remark{
	By solving \begin{equation}
	\frac{\partial f_p}{\partial p}=0
	\end{equation}, we get \begin{equation}
	\lim_{c\rightarrow 0}(a,b)=\left(\frac{k}{n},\frac{k}{n}\right),
	m=(n+1)\binom{n}{k}\left(\frac{k}{n}\right)^k\left(1-\frac{k}{n}\right)^{n-k}
	\end{equation}
}
\subsection{Table of small trial half probability intervals}
Table with $c=\frac{1}{2}$:\\
\begin{tabular}{ |c||c|c|c|c|c| }
	\hline
	& $k=0$ & $k=1$ & $k=2$ & $k=3$ & $k=4$ \\
	\hline
	$n=0$ & $\begin{array}{c}a \approx 0.2500000000\\b \approx 0.7500000000\end{array}$ \\
	\hline
	$n=1$ & $\begin{array}{c}a \approx 0.000000000\\b \approx 0.2928932188\end{array}$ & $\begin{array}{c}a \approx 0.7071067812\\b \approx 1.000000000\end{array}$ \\
	\hline
	$n=2$ & $\begin{array}{c}a \approx 0.000000000\\b \approx 0.2062994740\end{array}$ & $\begin{array}{c}a \approx 0.326351822\\b \approx 0.673648178\end{array}$ & $\begin{array}{c}a \approx 0.7937005260\\b \approx 1.000000000\end{array}$ \\
	\hline
	$n=3$ & $\begin{array}{c}a \approx 0.000000000\\b \approx 0.1591035847\end{array}$ & $\begin{array}{c}a \approx 0.1966713241\\b \approx 0.4921993002\end{array}$ & $\begin{array}{c}a \approx 0.507800700\\b \approx 0.803328676\end{array}$ & $\begin{array}{c}a \approx 0.8408964153\\b \approx 1.000000000\end{array}$ \\
	\hline
	$n=4$ & $\begin{array}{c}a \approx 0.000000000\\b \approx 0.1294494367\end{array}$ & $\begin{array}{c}a \approx 0.1387754454\\b \approx 0.3894044243\end{array}$ & $\begin{array}{c}a \approx 0.3594361648\\b \approx 0.640563835\end{array}$ & $\begin{array}{c}a \approx 0.610595575\\b \approx 0.861224555\end{array}$ & $\begin{array}{c}a \approx 0.8705505633\\b \approx 1.000000000\end{array}$ \\
	\hline
	$n=5$ & $\begin{array}{c}a \approx 0.000000000\\b \approx 0.1091012819\end{array}$ & $\begin{array}{c}a \approx 0.1065967785\\b \approx 0.3226910840\end{array}$ & $\begin{array}{c}a \approx 0.2770309506\\b \approx 0.532125995\end{array}$ & $\begin{array}{c}a \approx 0.4678740055\\b \approx 0.722969050\end{array}$ & $\begin{array}{c}a \approx 0.677308916\\b \approx 0.893403222\end{array}$ \\
	\hline
	$n=6$ & $\begin{array}{c}a \approx 0.000000000\\b \approx 0.09427633574\end{array}$ & $\begin{array}{c}a \approx 0.0862905298\\b \approx 0.2757178692\end{array}$ & $\begin{array}{c}a \approx 0.2249291772\\b \approx 0.4551185682\end{array}$ & $\begin{array}{c}a \approx 0.3788484407\\b \approx 0.621151559\end{array}$ & $\begin{array}{c}a \approx 0.544881432\\b \approx 0.775070823\end{array}$ \\
	\hline
	$n=7$ & $\begin{array}{c}a \approx 0.000000000\\b \approx 0.08299595680\end{array}$ & $\begin{array}{c}a \approx 0.07237321411\\b \approx 0.2407853463\end{array}$ & $\begin{array}{c}a \approx 0.1891248358\\b \approx 0.3976355162\end{array}$ & $\begin{array}{c}a \approx 0.3180913474\\b \approx 0.544067630\end{array}$ & $\begin{array}{c}a \approx 0.4559323705\\b \approx 0.681908652\end{array}$ \\
	\hline
	$n=8$ & $\begin{array}{c}a \approx 0.000000000\\b \approx 0.07412528771\end{array}$ & $\begin{array}{c}a \approx 0.06226629927\\b \approx 0.2137614499\end{array}$ & $\begin{array}{c}a \approx 0.1630534052\\b \approx 0.3530822174\end{array}$ & $\begin{array}{c}a \approx 0.2740192600\\b \approx 0.4838784786\end{array}$ & $\begin{array}{c}a \approx 0.3919638426\\b \approx 0.608036157\end{array}$ \\
	\hline
\end{tabular}
\begin{tabular}{ |c||c|c|c|c| }
	\hline
	& $k=5$ & $k=6$ & $k=7$ & $k=8$ \\
	\hline
	$n=5$ & $\begin{array}{c}a \approx 0.8908987181\\b \approx 1.000000000\end{array}$ \\
	\hline
	$n=6$ & $\begin{array}{c}a \approx 0.724282131\\b \approx 0.913709470\end{array}$ & $\begin{array}{c}a \approx 0.9057236643\\b \approx 1.000000000\end{array}$ \\
	\hline
	$n=7$ & $\begin{array}{c}a \approx 0.602364484\\b \approx 0.810875164\end{array}$ & $\begin{array}{c}a \approx 0.759214654\\b \approx 0.927626786\end{array}$ & $\begin{array}{c}a \approx 0.9170040432\\b \approx 1.000000000\end{array}$ \\
	\hline
	$n=8$ & $\begin{array}{c}a \approx 0.516121520\\b \approx 0.725980740\end{array}$ & $\begin{array}{c}a \approx 0.646917782\\b \approx 0.836946595\end{array}$ & $\begin{array}{c}a \approx 0.786238550\\b \approx 0.937733701\end{array}$ & $\begin{array}{c}a \approx 0.9258747123\\b \approx 1.000000000\end{array}$ \\
	\hline
\end{tabular}

\section{The second trial and consistency}
Suppose we had a first trial with $k_0$ events occurring out of $n_0$ chances to occur. Then the prior probability distribution would be:
\begin{equation}
f_{p}
=(n_{0}+1)\binom{n_{0}}{k_{0}}p^{k_{0}}(1-p)^{n_{0}-k_{0}}
\end{equation}
The second trial has $k_1$ events occurring out of $n_1$ chances to occur. By treating both trials as a single, larger trial we can jump ahead to get this equation:
\begin{equation}
f_{p|n_{1},k_{1}}
=(n_{0}+n_{1}+1)\binom{n_{0}+n_{1}}{k_{0}+k_{1}}
p^{k_{0}+k_{1}}(1-p)^{n_{0}+n_{1}-k_{0}-k_{1}}
\end{equation}
We will also show this the long way.


\end{document}

















