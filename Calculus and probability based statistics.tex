
\documentclass[8pt]{amsart}

\usepackage{amssymb}
\usepackage{amsmath}

\newtheorem{thm}{Theorem}[section]
\newtheorem{prop}[thm]{Proposition}
\newtheorem{lem}[thm]{Lemma}
\newtheorem{cor}[thm]{Corollary}


\theoremstyle{definition}
\newtheorem{definition}[thm]{Definition}
\newtheorem{example}[thm]{Example}


\theoremstyle{remark}
\newtheorem{remark}[thm]{Remark}

\numberwithin{equation}{section}

\newcommand{\R}{\mathbb{R}} % The real numbers.
\newcommand{\dd}[1]{\mathrm{d}#1}

\begin{document}

\title{Calculus and probability based statistics}

\author{Mark Andrew Gerads}
\date{\today}
\address{
318 Thomas Drive\\
Marshalltown, Iowa, 50158
}
\email{MGerads11@winona.edu\\nazgand@gmail.com}
\maketitle
\section{Finite statistics}
\subsection{Introduction}
The purpose of this text is to have statistics built on a solid foundation of probability and calculus.
\definition{$A$ is a repeatable event.}
\definition{$p$ is the probability that $A$ will happen when $A$ is given a chance to happen. For simplicity, we assume that $p$ is the same every time $A$ is given a chance to happen.}
\definition{$n\in\mathbb{W}$ is the number of times $A$ had a chance to happen.}
\definition{$k\in\mathbb{W}$ is the number of times $A$ actually happened.}
\remark{$k\leq n$.}
\definition{$f_p$ is the probability is the probability distribution for $p$.}
\remark{\label{IntegrateToUnity}$1=\int_0^1 f_p \dd p$}
\definition{$m$ is the maximum value of $f_p$.}
\definition{$(a,b)$ is the prediction interval of $p$ such that $b-a$ is minimized.}
\definition{$c$ is the desired certainty for the prediction interval of $p$.}
\remark{$0\leq c=\int_a^b f_p \dd p\leq 1$}
\remark{$0\leq a\leq b\leq 1$}


Now that we have the definitions, here is the problem: $n,k,c$ are known, and we must solve for $a,b$.
\remark{The probability of $k$ events given $n$ chances is $\binom{n}{k}p^k(1-p)^{n-k}$, but we must scale it to satisfy \ref{IntegrateToUnity}:
\begin{equation}
f_p
=\frac{\binom{n}{k}p^k(1-p)^{n-k}}{\int\limits_{0}^{1}\binom{n}{k}p^k(1-p)^{n-k}\dd p}
=\frac{p^k(1-p)^{n-k}}{\int\limits_{0}^{1}p^k(1-p)^{n-k}\dd p}
=(n+1)\binom{n}{k}p^k(1-p)^{n-k}
\end{equation}
}
\subsection{The 4 cases of finding the interval}

\subsubsection{Case $n=k=0$}
This case is self-symmetrical.
\remark{We have \begin{equation}
(a,b)=\left(\frac{1-c}{2},\frac{1+c}{2}\right)
\end{equation}}

\subsubsection{Case $n>k=0$}
\label{Symetry 1A}
This case is symmetrical to case \ref{Symetry 1B}.
\remark{We can solve \begin{equation}
\int_0^b (n+1)(1-p)^{n} \dd p=c=1-(1-b)^{n+1}
\end{equation}}
\remark{We have \begin{equation}
	(a,b)=\left(0,1-(1-c)^\frac{1}{n+1}\right)
	\end{equation}}

\subsubsection{Case $n=k>0$}
\label{Symetry 1B}
This case is symmetrical to case \ref{Symetry 1A}.
\remark{We can solve \begin{equation}
	\int_a^1 (k+1)p^k \dd p=c=1-a^{k+1}
	\end{equation}}
\remark{We have \begin{equation}
	(a,b)=\left((1-c)^\frac{1}{k+1},1\right)
	\end{equation}}

\subsubsection{Case $n>k>0$}
This case is self-symmetrical.
\remark{We desire to solve \begin{equation}
	\int_a^b (n+1)\binom{n}{k}p^k(1-p)^{n-k} \dd p=c=(n+1)\binom{n}{k}\left(B_b(k+1,-k+n+1)-B_a(k+1,-k+n+1)\right)
	\end{equation} where $B$ is the incomplete beta function.
}
\remark{
 We can use the hint
	\begin{equation}
	a^k(1-a)^{n-k}=b^k(1-b)^{n-k}
	\end{equation}
}
\remark{
	By solving \begin{equation}
	\frac{\partial f_p}{\partial p}=0
	\end{equation}, we get \begin{equation}
	\lim_{c\rightarrow 0}(a,b)=\left(\frac{k}{n},\frac{k}{n}\right),
	m=(n+1)\binom{n}{k}\left(\frac{k}{n}\right)^k\left(1-\frac{k}{n}\right)^{n-k}
	\end{equation}
}
\end{document}

















